\documentclass[a4paper,10pt]{article}
\usepackage[utf8]{inputenc}
\usepackage{glossaries}
\usepackage{fullpage}
\usepackage{hyperref}
\usepackage{amsmath}
\setlength\parindent{0pt}
\setlength{\parskip}{10pt}

\makeglossaries
\newglossaryentry{vlbi}{%
name={VLBI},%
description={Very Long Baseline Interferometry}%
}
\newglossaryentry{fir}{%
name={FIR},%
description={Finite Impulse Response}%
}
\newglossaryentry{pfb}{%
name={PFB},%
description={Polyphase Filter Bank}%
}
\newglossaryentry{lo}{%
name={lo},%
description={Local Oscillator}%
}
\newglossaryentry{pr}{%
name={PR},%
description={Perfect Reconstruction}%
}
\newglossaryentry{fpga}{%
name={FPGA},%
description={Field Programmable Gate Array}%
}

\begin{document}
\begin{titlepage}

\newcommand{\HRule}{\rule{\linewidth}{0.5mm}} % Defines a new command for the horizontal lines, change thickness here

\center % Center everything on the page
 
%----------------------------------------------------------------------------------------
%	HEADING SECTIONS
%----------------------------------------------------------------------------------------

\textsc{\LARGE Square Kilometer Array South Africa}\\[1.5cm]
\textsc{\Large Internship technical report}\\[0.5cm]

%----------------------------------------------------------------------------------------
%	TITLE SECTION
%----------------------------------------------------------------------------------------

\HRule \\[0.4cm]
{ \huge \bfseries KAT-7 GPU-accelerated Inverse Polyphase Filterbank}\\[0.4cm]
\HRule \\[1.5cm]
 
%----------------------------------------------------------------------------------------
%	AUTHOR SECTION
%----------------------------------------------------------------------------------------

\begin{minipage}{0.4\textwidth}
\begin{flushleft} \large
\emph{Author:}\\
Benjamin \textsc{Hugo}\\ % Your name
\small{bennahugo@aol.com}
\end{flushleft}
\end{minipage}
~
\begin{minipage}{0.4\textwidth}
\begin{flushright} \large
\emph{Advisor:} \\
Ludwig Schwardt
\end{flushright}
\end{minipage}\\[4cm]

% If you don't want a supervisor, uncomment the two lines below and remove the section above
%\Large \emph{Author:}\\
%John \textsc{Smith}\\[3cm] % Your name

%----------------------------------------------------------------------------------------
%	DATE SECTION
%----------------------------------------------------------------------------------------

{\large \today}\\[3cm] % Date, change the \today to a set date if you want to be precise
{The results obtained in this report was made possible through the use of the ICTS High Performance HEX cluster 
of the University of Cape Town (UCT) \url{hex.uct.ac.za}. The author wishes to thank Andrew Lewis of UCT ICTS for 
the technical support he provided during this project.}
%----------------------------------------------------------------------------------------
%	LOGO SECTION
%----------------------------------------------------------------------------------------

%\includegraphics{Logo}\\[1cm] % Include a department/university logo - this will require the graphicx package
 
%----------------------------------------------------------------------------------------

\vfill % Fill the rest of the page with whitespace

\end{titlepage}

\printglossary[style=long]
\pagebreak
\tableofcontents
\pagebreak
\section{Introduction}
At its core Very Long Baseline Interferometry (\gls{vlbi}) is an interferometry process where the output from several radio antennae are 
combined to form an equivalent output to a telescope of the size equal to the distance between the furthest two antennae in the \gls{vlbi} array.
For a comprehensive overview of the technique the reader is referred to the detailed introduction by Middelberg et al. \cite{middelberg2008high}. Several
such extended arrays exist including the European VLBI network and the Very Long Baseline Array. The hope is to include the KAT-7 in future VLBI
observations.

Traditionally the process required raw data to be dumped to a storage medium, say tape and physically shipped to a correlator where the data from several 
telescopes could be combined. The new trend in VLBI is to perform real-time correlation between antennae using high-speed internet connections and is known
as 'eVLBI'. Both the Australian Long Baseline Array and the European VLBI network has performed successful eVLBI observations.

Ultimately the problem being investigated (at least in part by this report) can be boiled down to converting data sent over the SPEAD protocol (employed 
internally by the KAT-7 array) to the VDIF format. This conversion process includes a necessary first step where the current sampling rate of 800 MHz
\footnote{According to the KAT-7 Data Interface document} is reduced to 128 MHz through \textit{Digital Downconversion}. The basic operation involves the following three
steps:
\begin{enumerate}
 \item Mixing. Where the signal is shifted down to \textit{baseband} (frequency 0 of the spectrum), by mixing the signal with a tone produced by a Local Oscillator (\gls{lo}).
 This is simply a generated sine wave at the lower end of the sub-spectrum to be extracted. This is simply done by an element-wise multiplication of the original signal with
 the local oscillator tone. In essence mixing is not a \textit{linear} (refer to \cite[ch. 5]{smith1997scientist}) process. If $f$ was a single channel in the 
 frequency domain then mixing produces replicated channels at $f - f_{lo}$ and $f + f_{lo}$.
 \item Filtering. In order to eliminate aliasing in the frequency domain due to mixing and frequencies above the new sampling rate we use a Finite Impulse Response 
 (\gls{fir}) filter with the cutoff set at the rate $\frac{1}{2}f_{decimated}$ to comply with the Nyquest sampling theorem (see \cite[ch. 3]{smith1997scientist}).
 \item Interpolation and Decimation. The reader is referred to \url{http://www.dspguru.com/dsp/faqs/multirate/basics} for an overview of the process.  
\end{enumerate}

There is, however, a further complication to deal with, before this downconversion process. The KAT-7 beamformer produces a series of frequency spectra\footnote{Each 
with 1024 channels according to the KAT-7 Data Interface}. If no filtering was applied these spectra the undo operation would only have involved applying the inverse
Discrete Fourier Transform. However, the original voltage data went through a filter-bank operation in a process known as the Polyphase Filter Bank (\gls{pfb}). This
method is also known as the Weighted Window Overlap method. The reader is referred to \url{https://casper.berkeley.edu/wiki/The_Polyphase_Filter_Bank_Technique} for a
detailed description on this technique.

\section{Method}
The forward \gls{pfb} process can be thought of as a very basic analysis filter bank, whereas the inverse process can be thought of as a synthesis filter bank. The 
analysis filter bank uses a Hamming-windowed FIR with $P=8$ banks of $N=1024$ elements, with no up or down-sampling stages \footnote{According to information provided by Jason Manley}. 
The key differences between the analysis and synthesis processes are:
\begin{enumerate}
 \item If $H[n]$ is the prototype FIR filter and the analysis filter bank uses the subfilters $H_1[n], H_2[n] \dots H_P[n]$ each of length $N$ then the synthesis filter bank
 uses the subfilters $\bar H_1[N-n-1], \bar H_2[N-n-1] \dots \bar H_P[N-n-1]$\footnote{Findings published in an in-house final technical report titled 'A review of polyphase filter 
 banks and their application'. Daniel Zhou. Air Force Research Laboratory, Information Directorate, Rome Research Site, Rome, New York.}
 \item The commutator is flipped around (as shown on page 10 of Zhou's report) and therefore the sub filters should be processed in reverse order.
\end{enumerate}

According to Parishwad Vaidyanathan \cite{vaidyanathan1990multirate} this basic filter bank cannot achieve Perfect Reconstruction (\gls{pr}) and have to be modified in order to
do so. This involves modifying both the analysis and synthesis filter banks. In our situation this is not plausible, since the only the basic form of an analysis filter bank
is implemented on Field Programmable Gate Arrays (\gls{fpga}) due to their simple construction and fast evaluation times \cite{vaidyanathan1990multirate}. Vaidyanathan goes on
to state that the synthesis filter bank requires subfilters of much greater length to obtain reasonably good reconstruction and is therefore computationally prohibiting.

\pagebreak
\bibliography{inverse_polyphase_filter_bank_report}{}
\bibliographystyle{plain}
\end{document}          
